\documentclass[UTF8,a4paper,zihao=-4]{article}
\usepackage{xeCJK}
\usepackage{fontspec}
\usepackage{fontsize}
\usepackage{amsmath}
\usepackage{biblatex}
\usepackage{geometry}
\usepackage[normalem]{ulem}
\usepackage{amssymb}
\xeCJKsetup{CJKmath=true}
\newcommand\funderline{\bgroup\markoverwith
   {%
     \rule[-1ex]{2pt}{0.4pt}%
   }%
   \ULon}
\newcommand\sunderline{\bgroup\markoverwith
   {%
     \rule[-1ex]{2pt}{0pt}%
     \llap{{\rule[-2.7ex]{2pt}{0.4pt}}}%
   }%
   \ULon}
\newcommand\tunderline{\bgroup\markoverwith
   {%
     \rule[-1ex]{2pt}{0pt}%
     \llap{{\rule[-2.7ex]{2pt}{0pt}}}%
     \llap{{\rule[-4.4ex]{2pt}{0.4pt}}}%
   }%
   \ULon}
\newcommand\fsunderline{\bgroup\markoverwith
   {%
     \rule[-1ex]{2pt}{0.4pt}%
     \llap{{\rule[-2.7ex]{2pt}{0.4pt}}}%
   }%
   \ULon}
\newcommand\ftunderline{\bgroup\markoverwith
   {%
     \rule[-1ex]{2pt}{0.4pt}%
     \llap{{\rule[-2.7ex]{2pt}{0pt}}}%
     \llap{{\rule[-4.4ex]{2pt}{0.4pt}}}%
   }%
   \ULon}
\newcommand\stunderline{\bgroup\markoverwith
   {%
     \rule[-1ex]{2pt}{0pt}%
     \llap{{\rule[-2.7ex]{2pt}{0.4pt}}}%
     \llap{{\rule[-4.4ex]{2pt}{0.4pt}}}%
   }%
   \ULon}
\newcommand\fstunderline{\bgroup\markoverwith
   {%
     \rule[-1ex]{2pt}{0.4pt}%
     \llap{{\rule[-2.7ex]{2pt}{0.4pt}}}%
     \llap{{\rule[-4.4ex]{2pt}{0.4pt}}}%
   }%
   \ULon}
\title{筛法}
\begin{document}
\author{YeongJet.Tang}
\date{}
\maketitle
筛法是用于求不超过某个自然数的全部质数的方法, 最古老的筛法是由古希腊的埃拉托斯特尼(Eratosthenes)发明的,又称埃拉托斯特尼筛法。该筛法简要描述如下: 要得到自然数n以内的全部质数, 把不大于根号n的所有素数的倍数剔除, 剩下的就是$\sqrt{x}$和$x$之间的质数。例如, 我们要筛选出30以内的质数\\
质数的定义是该数的因数只有1和它本身,例如3的因数是1和3(它本身), 那么,如何求小于某个正整数的全部质数? 假设我们要筛选出30以内的全部质数,该怎么做呢?
首先向大家说一个事实,假设我们有某个质数$p$,那么把大于$p$并且小于$p^2$的数删去,
我们把30以内的自然数列出来: 0,1,2,3,...,28,29,30
2是最小的质数,把凡是是2的倍数的数(2,4,6,8,...,28,30)删去,为了后续方便向大家说明,我们在这些数字下面划了一条横线表示该数是2的倍数, 并且在最有端用2表示该数是2的倍数这样留下来的第一个数必定是素数,因为除了1以外,它不能被小于它的数整除,也不能
我们注意到,某些数下面不止一条横线,说明该数被删了多次,例如6能同时被2和3删去,所以有两条横线,横线所在行最左边,我们用质数2,3,5表示该行画横线的
第一行的横线表示该横线上的数能被2删去,第二行的横线表示该横线上的数能被3删去,如此类推
\begin{align*}
  \sigma
    &\underset{5}{\underset{3}{\underset{2}{\ }}}\hspace{0.1em}\fstunderline{\ 0\ }\hspace{0.9em}{\ 1\ }\hspace{0.9em}\funderline{\ 2\ }\hspace{0.8em}\sunderline{\ 3\ }\hspace{0.8em}\funderline{\ 4\ }\hspace{0.8em}\tunderline{\ 5\ }\hspace{0.8em}\fsunderline{\ 6\ }\hspace{0.8em}{\ 7\ }\hspace{0.8em}\funderline{\ 8\ }\hspace{0.8em}\sunderline{\ 9\ }\quad\ftunderline{10}\quad11\quad\fsunderline{12}\quad13\quad\funderline{14}\quad\stunderline{15}\\\\
    &\underset{5}{\underset{3}{\underset{2}{\ }}}\hspace{0.2em}\funderline{16}\quad17\quad\fsunderline{18}\quad19\quad\ftunderline{20}\quad\sunderline{21}\quad\funderline{22}\quad23\quad\fsunderline{24}\quad\tunderline{25}\quad\funderline{26}\quad\sunderline{27}\quad\funderline{28}\quad29\quad\fstunderline{30}
\end{align*}
\end{document}