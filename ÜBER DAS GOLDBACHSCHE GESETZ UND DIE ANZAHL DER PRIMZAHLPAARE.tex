\documentclass{article}
\begin{document}
§1

Eine sehr interessante Aufgabe der Primzahltheorie ist die folgende: Gibt es, trotz der abnehmenden Häufigkeit der Primzahlen, doch unendlich viele Paare von Primzahlen mit Differenz 2, wie 11-13, 17-19 und so weiter?
Es ist noch niemand gelungen, etwas darüber zu beweisen, und so viel ich weiss, hat auch niemand versucht eine Näherungsformel über die Anzahl der Primzahlpaare unter x aufzustellen.
Ich bin darauf aufmerksam geworden, dass die Methode , die Eratosthenes zur Auffindung der Primzahlen benutzte, auch zur Auffindung der Primzahlpaare fähig ist.
Durch Überlegungen über diese Methode bin ich dazu geführt, eine Näherungsformel über die Anzahl $(p(x))$ der Primzahlpaare unter $x$ aufzustellen , nämlich$$ p(x)=k\frac{x}{log^2x}\tag2\hspace{7em}$$wo $k$ eine Konstante bedeutet.
Ich bin auch darauf aufmerksam geworden, dass die Methode Eratosthenes's fähig ist die Primzahlen auszusieben , deren Summe eine gegebene Zahl ist. Goldbach hat folgende Vermutung ausgesprochen : « Jede gerade Zahl lässt sich als Summe. zweier Primzahlen schreiben. » Euler bemerkt im Jahre 1742 dazu : « Ich halte dies für ein ganz gewisses theorema, ungeachtet ich dasselbe nicht demonstrieren kann. »
Durch ähnliche Überlegungen wie die früher genannten bin ich dazu geführt folgende Näherungsformel aufzustellen :
Eine gerade Zahl $x$ lässt sich in $G(x)$ verschiedenen Weisen als Summe von zwei Primzahlen schreiben, wo$$G(x)=k\cdot\frac{x}{log^2x}\cdot\frac{p-1}{p-2}\cdot\frac{q-1}{q-2}\cdot....\cdot\frac{r-1}{r-2}$$
Hier bedeuten $p, q,...r$ die ungeraden Primfaktoren in $x$, die kleiner als $\sqrt{x}$ sind, während $k$ dieselbe Konstante wie in Formel $Ⅰ$ bedeutet. Wir haben dann zwischen den Decompositionen $p_1+p_2$ und $p_2 +p_1$ gesondert.
Stäckel hat im Jahre 1896*) eine ähnliche Näherungsformel aufgestellt. Seine Formel lässt sich folgendermassen schreiben:$$G(x)=2\frac{x}{log^2x}\cdot\frac{p}{p-1}\cdot\frac{q}{q-1}\cdot....\cdot\frac{r}{r-1}$$
Stäckel hat sich einer Wahrscheinlichkeitsbetrachtung und der Entdeckung benutzt, dass die Schwankungen von G(x) mit den Schwankungen von$$\varphi(x)=x\cdot\frac{p-1}{p}\cdot\frac{p-1}{p}\cdot...\cdot\frac{p-1}{p}$$
in Zusammenhang stehen.
Landau hat im Jahre 1900**) die Stäckelsche Formel näher behandelt und hat vorgeschlagen, die Formel mit der Konstante 0,772 zu multiplizieren . Wie man sieht, ist die Formel II wesentlich verschieden von der Stäckel- Landauschen.
Die Stäckelsche Wahrscheinlichkeitsbetrachtung ist auch nach meiner Meinung nicht befriedigend, indem er sich Elemente, deren Stellung sehr abhängig von einander ist, in irgend eine Reihenfolge » denkt (Seite 295).
Es ist mir nicht gelungen, die Näherungsformeln zu beweisen. Wenn ich trotzdem die Überlegungen, die dazu geführt haben, veröffentliche, ist es in der Hoffnung, dass sie den richtigen Weg zum Beweise der zwei Theoremen zeigen werden: Es gibt unendlich viele Primzahlpaare, und: Jede gerade Zahl kann als Summe von zwei Primzahlen dargestellt werden.

$$§2.$$

Die Siebmethode von Eratosthenes ist wohlbekannt. Wir werden sie in folgender Weise wiedergeben: Man wünscht die Primzahlen zwischen $\sqrt{x}$ und $x$ auszusieben. Wir wählen zum Beispiel $x=30$, wonach $\sqrt{x}=5$,... Wir schreiben die Zahlenreihe von $0$ bis $30$ auf:
$$0\quad1\quad2\quad3\quad4\quad5\quad6\quad7\quad8\quad9\quad10\quad11\quad12\quad13\quad14\quad15\quad16\quad17\quad18\\\quad19\quad20\quad21\quad22\quad23\quad24\quad25\quad26\quad27\quad28\quad29\quad30$$
Erstens streichen wir die durch 2 teilbaren Zahlen, * ) 0, 2, 4, 6 ..., also jede zweite Zahl, dann die durch 3 teilbaren Zahlen 0, 3, 6, 9 ..., oder jede dritte Zahl, indem wir auch. früher gestrichene mit einem neuen Strich versehen. Dann streicht man weiter jede fünfte Zahl, 0 , 5 , 10 ... Die zurückstehenden nicht gestrichenen Zahlen sind 1 und die Primzahlen zwischen 130 und 30. Wären sie zusammengesetzt, müssten sie ja durch eine Primzahl kleiner als 30 teilbar sein, also durch 2, 3 oder 5. Ganz allgemein genügt es mit den Primzahlen kleiner als V Streichungen durchzuführen. Wir werden jetzt eine ähnliche Methode anwenden um die Primzahlpaare auszusieben. Wir richten die Aufmerksamkeit auf die Zahl z, die zwischen den zwei Primzahlen des Paares belegen ist. Wir suchen also die Zahlen z, die folgende Eigenschaft haben:
$$
z+1=Primzahl\\
z-1=Primzahl
$$
Die Zahlen z, die die erste Eigenschaft besitzen, kann man leicht aussieben, indem man nur alle Streichungen einen Schritt nach links verschiebt. Statt bei 0 fangen wir also bei -1 an. In gleicher Weise sieben wir die Zahlen aus, die die zweite Eigenschaft besitzen , indem wir +1 als Ausgangspunkt der Streichungen benützen. Als Beispiel wählen wir $x=32$
$$-1\quad0\quad1\quad2\quad3\quad4\quad5\quad6\quad7\quad8\quad9\quad10\quad11\quad12\quad13\quad14\quad15\quad16\quad17\quad18\\\quad19\quad20\quad21\quad22\quad23\quad24\quad25\quad26\quad27\quad28\quad29\quad30\quad31\quad32$$
Die zurückstehenden Zahlen sind 12 , 18 und 30. Die entsprechenden Primzahlpaare sind $11-13$, $17-19$ und $29-31$.
Eine Aussiebung der Primzahlpaare zwischen $\sqrt{x}$ und $x$ kann man also erreichen durch eine zweimalige Anwendung der Methode Eratosthenes's, wenn man die Ausgangspunkte $-1$ und $+1$ statt $0$ wählt, und indem man jede zweite, jede dritte, jede fünfte bis jede pte Zahl unterstreicht, wenn $p$ die grösste Primzahl unter $x$ bedeutet.
Wir werden auch zeigen wie eine gerade Zahl $x$ in ähnlicher Weise in eine Summe von zwei Primzahlen dekomponiert werden kann. Wählen wir zum Beispiel $x=26$. Wir schreiben die Zahlen 0 bis 26 auf und unter ihnen in umgekehrter Reihenfolge dieselben Zahlen . Die Summe einer Zahl der ersten Reihe und der entsprechenden Zahl der zweiten Reihe ist dann immer $26$. Jetzt wenden wir Eratosthenes's Methode auf beide Reihen an. Die zurückstehenden Zahlenpaare in welchen die beiden direkt unter einander stehenden Zahlen nicht gestrichen sind, sind dann die gesuchten Primzahlen zwischen $\sqrt{x}$ und $x-\sqrt{x}$.
$$0\quad1\quad2\quad3\quad4\quad5\quad6\quad7\quad8\quad9\quad10\quad11\quad12\quad13\quad14\quad15\quad16\quad17\quad18\\\quad19\quad20\quad21\quad22\quad23\quad24\quad25\quad26\quad26\quad25\quad24\quad23\quad22\quad21\quad20\quad19\quad18\quad17\quad16\quad15\quad14\quad13\quad12\quad11\quad10\quad9\quad8\quad7\quad6\quad5\quad4\quad3\quad2\quad1\quad0$$
Die nicht gestrichenen Zahlenpaare sind $7$, $19$ und $13$, $13$ und $19$, $7$, deren Summe 26 ist. Selbstverständlich ist es unnötig die zweite Reihe von Zahlen aufzuschreiben. Man braucht ja nur die Striche dieser Reihe nach oben verschieben. Wir können mit anderen Worten die zweite Serie von Streichungen bei 26 anfangen und nach links durchführen.
Die Zerlegung einer geraden Zahl $x$ in die Summe von zwei Primzahlen zwischen $\sqrt{x}$ und $x-\sqrt{x}$ lässt sich ausführen durch eine zweimalige Anwendung der Methode Eratosthenes's, wenn man die Ausgangspunkte $0$ und $x$ wählt, und indem man jede zweite, jede dritte, jede fünfte bis jede $p$ te Zahl streicht, wenn $p$ die grösste Primzahl unter $\sqrt{x}$ bedeutet.
Da $x$ als gerade vorausgesetzt ist, sieht man, dass die zweimalige Streichung von jeder zweiten nur einer einmaligen Streichung entspricht. Man sieht auch, dass wenn $x$ durch $3$ teilbar ist, dann wird die zweimalige Streichung von jeder dritten Zahl nur einer einmaligen entsprechen. Es ist dann zu erwarten, dass die Anzahl der Zerlegungen relativ gross wird jedesmal $x$ durch $3$ teilbar ist. Dass dies auch der Fall ist, sieht man sogleich aus den Tabellen, die über die Goldbachschen Zerlegungen ausgearbeitet sind.

$$§3.$$
Die Methode Eratosthenes's ist auch dazu fähig die Primzahlen einer arithmetischen Reihe auszusieben . In einer arithmetischen Reihe
$$a\quad a + p\quad a + 2p\quad a + 3p\ ....$$
wo $p$ eine Primzahl bedeutet, ist jede zweite Zahl durch $2$ teilbar, wenn nicht eben $p = 2$ ist, und jede dritte Zahl durch $3$ teilbar, wenn nicht eben $p = 3$ ist u. s. w.
Wünschen wir z. B. die Primzahlen der Reihe$$3\quad 8\quad 13\quad 18\quad 23\quad 28\quad 33\quad 38\quad 43\quad 48$$auszusieben, müssen wir jede zweite und jede dritte Zahl streichen, mit den zwei verschiedenen Ausgangspunkten 8 und 3. Dies ist von besonderem Interesse bei unseren Siebmethoden.
Denken wir uns, dass wir erstens jede zweite und jede dritte Zahl der natürlichen Zahlenreihe mit willkürlichen Ausgangspunkten gestrichen haben, und dass wir dann jede fünfte Zahl streichen. Wie viele von diesen letzten sind dann früher gestrichen? Die Antwort wird: jede zweite und jede dritte, doch mit unbestimmten Ausgangspunkten. Dasselbe gilt bei doppelten Streichungen.
Es wird daher dienlich sein , unserm Problem folgende allgemeine Form zu geben:
Folgende arithmetische Reihen seien gegeben:$$\triangle\quad \triangle+D\quad \triangle+2D\quad \triangle+3D\quad.... a_1\quad a_1+2\quad a_1+4\quad a_1+6\quad ....a_2\quad a_2+3\quad b_2+2 b₂2 + 3 az az +5 2 as +10 a , +15 α₂ + 6 α₂2 + 9 b₂ + 6 b₂2 + 9 b3 b3 +5 b3g +10 bg3 +15 (A) (B) an an+pn an +2pn an +3pn bn bn+Pn bn +-2pn bn +3pn$$Sämtliche Reihen erstrecken sich von $0$ bis $x$, das heisst, das erste Glied ist kleinstmöglich positiv und das letzte Glied ist grösstmöglich kleiner als $x$z oder gleich $x$. Wir setzen noch nicht voraus, dass $p_n$ die grösste Primzahl unter $\sqrt{x}$ bedeutet. Wir denken uns weiter, dass $a_n b_n$ für alle $n$, was einer zweimaligen Streichung entspricht. $D$ ist eine ganze Zahl, die relativ prim zu $2,3,5,...p_n,p_n+1$ ist.
Unser Problem wird das folgende:
Wie viele Glieder enthält die erste Reihe $A$, die keinem Gliede der untenstehenden Reihen gleich sind?
Wir werden diese Anzahl mit $P_{p_n}(x)$ bezeichnen und werden beweisen, dass$$P_{p_n}(x)=\frac{x}{D}(1-\frac{1}{2})(1-\frac{2}{3})(1-\frac{2}{5})...(1-\frac{2}{p_n})+\Theta\cdot3^n$$wo$$-1<\Theta<1$$Die Formel enthält ein unbestimmtes Restglied. Die Aufgabe ist ja auch teilweise unbestimmt, indem die Ausgangspunkte der Streichungen, das heisst die Grössen $a$ und $b$, willkürlich wählbar sind.
Wir setzen voraus, dass:$$P_{p_n}(x)=K_n\frac{x}{D}+\Theta R_n$$und wir suchen $P_{p_{n+1}}(x)$ zu bestimmen.
Wir fügen dann folgende neue Reihen zu den vorigen:$$an+1 an+1 +P +1 an + 1 +2pn+ 1 an+1 +3p + 1 bu+1 bn+1 +P +1 ba + 1 + 2pn+ 1 bu+1 +3Pn+ 1 (C) (D)$$
die sich auch zwischen 0 und x erstrecken . Wir haben dann zu untersuchen, wie viele von diesen Gliedern folgende zwei Bedingungen erfüllen : identisch mit einem Gliede der Reihe A, aber verschieden von sämtlichen Gliedern der Reihen B zu sein.
Die identischen Glieder der Reihen A und C und der Reihen A und D sind die Glieder folgender zwei arithmetischen Reihen:$$a a +Pn+1 D b b + Pn+ 1D b + 2pn+ 1 D b + 3Pn+ 1 D a + 2pn+1D a + 3pn+1 D$$
die sich auch zwischen 0 und x erstrecken und wo a b. Die Anzahl von Gliedern in diesen, die von sämtlichen Gliedern der Reihen B verschieden sind, wird dann gleich $2\left(K_n\frac{x}{D_{p_n}}+\Theta R_n\right)$ wodurch:
$$P_{p_n}(x)=K_n\frac{x}{D}+\Theta R_n-2\left(K_n\frac{x}{D_{p_{n+1}}}+\Theta R_n\right)$$
oder$$P_{p_n}(x)=K_n(1-\frac{2}{p_{n+1}})\frac{x}{D}+\Theta 3R_n$$Da nun$$P_2(x)=\frac{1}{2}\frac{x}{D}+\Theta\cdot3$$wird$$P_3(x)=\frac{1}{2}\left(1-\frac{2}{3}\right)\frac{x}{D}+\Theta\cdot3^2$$und schliesslich$$P_{p_n}(x)=\left(1-\frac{1}{2}\right)\left(1-\frac{2}{3}\right)\left(1-\frac{2}{5}\right)...\left(1-\frac{2}{p_n}\right)\frac{x}{D}+\Theta\cdot3^n$$
Wenn $D=1$ ist, wenn also die Reihe $A$ aus den Zahlen $1 ,2,3,... x$ besteht, erhalten wir die Formel:$$P_{p_n}(x)=\left(1-\frac{1}{2}\right)\left(1-\frac{2}{3}\right)\left(1-\frac{2}{5}\right)...\left(1-\frac{2}{p_n}\right)x+\Theta\cdot3^n$$Ist die Anzahl der gegebenen arithmetischen Reihen von $x$ unabhängig, ist also das Problem einfach. Dann gilt die asymptotisch richtige Formel$$P_{p_n}(x)=x\left(1-\frac{1}{2}\right)\left(1-\frac{2}{3}\right)\left(1-\frac{2}{5}\right)...\left(1-\frac{2}{p_n}\right)$$Hat man mit einer einfachen Streichung zu tun, wie bei der Methode Eratosthenes's, erhält man selbstverständlich die Näherungsformel
\end{document}